% backup -

\documentclass[sigconf]{acmart}
\settopmatter{printacmref=true}
\fancyhead{}
\usepackage{algorithm, pseudocode}
\usepackage{booktabs} 
\usepackage{graphicx}
%\usepackage{amssymb}
\usepackage{amsfonts}
\usepackage{amsmath}
\usepackage{amsthm}
\usepackage{bm}
\usepackage{etex}
\usepackage{color}
\usepackage{graphicx}
\usepackage{esint}
%\usepackage{stmaryrd}
\usepackage{tabularx}
\usepackage{multirow}
%\usepackage[squaren]{SIunits}
\usepackage[tight]{subfigure}
\usepackage{multirow}
\usepackage{mathtools}
\usepackage[noend]{algpseudocode}
\usepackage[colorinlistoftodos]{todonotes}
\usepackage{chngcntr}
\usepackage{titlesec}

\newtheorem{definition}{Definition}
\newtheorem{theorem}{Theorem}
\newtheorem{lemma}{Lemma}
\newtheorem{claim}{Claim}
\newtheorem{sketch}{Sketch of Proof}
\newtheorem{example}{Example}
\newtheorem{problem}{Problem}
\newtheorem{prop}{Proposition}

\newcommand{\M}{\mathsf{M}}

\newcommand{\A}{\mathbb{A}}
\newcommand{\Q}{\mathbb{Q}}
\renewcommand{\P}{\mathbb{P}}
\newcommand{\K}{\mathbb{K}}
\newcommand{\F}{\mathbb{F}}
\newcommand{\Z}{\mathbb{Z}}
\newcommand{\N}{\mathbb{N}}
\renewcommand{\L}{\mathbb{L}}
\newcommand{\ang}[1]{\{#1\}}
\newcommand{\mb}{\overline{\mathcal{M}}}
\newcommand{\mm}{\mathcal{M}}
\newcommand{\ee}{\mathcal{L}}
\newcommand{\kk}{\mathcal{K}}
\newcommand{\lm}{\textnormal{lm}}
\newcommand{\rr}{\mathcal{R}}
\newcommand{\pp}{\mathcal{P}}
\newcommand{\ints}{\mathbb{Z}}
\newcommand{\cN}{\mathcal{N}}
\newcommand{\ar}{\mathcal{A}}
\newcommand{\cO}{\mathcal{O}}
\newcommand{\la}{\left\langle}
\newcommand{\ra}{\right\rangle}
\newcommand{\red}{\textnormal{red}}
\newcommand{\cor}{\textnormal{Corr}}
\newcommand{\pe}{\textnormal{Per}}
\newcommand{\inn}{\textnormal{Inn}}
\newcommand{\reg}{\textnormal{reg}}
\newcommand{\supp}{\textnormal{supp}}
\newcommand{\mut}{\textnormal{MutNorm}}
\newcommand{\bas}{\textnormal{base}}
\newcommand{\ngen}{\textnormal{NormGen}}
\newcommand{\intt}{\textnormal{Int}}
\newcommand{\gen}{\textnormal{gen}}
\newcommand{\norm}{\textnormal{norm}}
\newcommand{\maxn}{\textnormal{MaxNorm}}
\newcommand{\act}{\textnormal{act}}
\newcommand{\aff}{\mathbb{A}}
\newcommand{\affn}{\mathbb{A}^n}
\newcommand{\spa}{\textnormal{ }}
\newcommand{\lcm}{\textnormal{lcm}}
\newcommand{\divi}{\textnormal{div}}
\newcommand{\Reg}{\textnormal{Reg}}
\newcommand{\spec}{\textnormal{Spec}}
\newcommand{\conv}{\textnormal{Conv}}
\newcommand{\cone}{\textnormal{Cone}}
\newcommand{\minpol}{\textnormal{MinPoly}_{\mathbb{F}_q}}
\newcommand{\modu}{\textnormal{ mod }}
\newcommand{\frakf}{\mathfrak{f}}
\newcommand{\frakp}{\mathfrak{p}}
\newcommand{\frakr}{\mathfrak{r}}
\newcommand{\softO}{O\tilde{~}}
\newcommand{\sring}{\L\{\tau\}}
\newcommand{\hatr}{\hat{r}}
\newcommand{\reduc}{\lambda}
\newcommand{\comm}{\mathbb{F}_q[\tau^n]}
\newcommand{\sqdet}{\det}
\DeclarePairedDelimiter\ceil{\lceil}{\rceil}
\DeclarePairedDelimiter\floor{\lfloor}{\rfloor}
% \newcommand{\todo}{{\bf todo:}~}

\setcopyright{rightsretained}


% DOI
%% \acmDOI{10.475/123_4}
% ISBN
%% \acmISBN{123-4567-24-567/08/06}
%Conference
%% \acmConference[WOODSTOCK'97]{ACM Woodstock conference}{July 1997}{El Paso, Texas USA}
%% \acmYear{2019}
%% \copyrightyear{2019}
%% \acmArticle{4}
%% \acmPrice{15.00}


%\acmDOI{?/?}
%\acmISBN{?}
%\acmConference[ISSAC]{International Symposium on Symbolic and Algebraic Computation}{2019}{Beijing}
%\acmYear{2019}
%\copyrightyear{2019}
%\acmPrice{??.??}

\begin{document}
\copyrightyear{2019} 
\acmYear{2019} 
\setcopyright{acmlicensed}
\acmConference[ISSAC '19]{International Symposium on Symbolic and Algebraic Computation}{July 15--18, 2019}{Beijing, China}
\acmBooktitle{International Symposium on Symbolic and Algebraic Computation (ISSAC '19), July 15--18, 2019, Beijing, China}
\acmPrice{15.00}
\acmDOI{10.1145/3326229.3326256}
\acmISBN{978-1-4503-6084-5/19/07}
\title{Minimal Polynomials of Finite Drinfeld Modules in any Rank}

\author{Yossef Musleh}
\affiliation{%
  \institution{Cheriton School of Computer Science \\ University of Waterloo}
  \city{Waterloo}
  \state{Ontario}
  \country{Canada}
}
\email{ymusleh@uwaterloo.ca}

\author{\'Eric Schost}
\affiliation{%
  \institution{Cheriton School of Computer Science \\ University of Waterloo}
    \city{Waterloo}
  \state{Ontario}
  \country{Canada}
}
\email{eschost@uwaterloo.ca}



\begin{abstract}

\end{abstract}


\setcounter{secnumdepth}{4}
\renewcommand{\theparagraph}{\thesubsection.\arabic{paragraph}}
\counterwithin{paragraph}{subsection} % makes paragraph depend on subsection
\titleformat{\paragraph}[runin]{\normalfont\normalsize\bfseries}{\theparagraph.}{1em}{}
\titlespacing*{\paragraph}{0em}{1ex}{1em}
\newcommand{\pref}[1]{{\bf\ref{#1}}}

\maketitle

%%%%%%%%%%%%%%%%%%%%%%%%%%%%%%%%%%%%%%%%%%%%%%%%%%%%%%%%%%%%
%%%%%%%%%%%%%%%%%%%%%%%%%%%%%%%%%%%%%%%%%%%%%%%%%%%%%%%%%%%%
%%%%%%%%%%%%%%%%%%%%%%%%%%%%%%%%%%%%%%%%%%%%%%%%%%%%%%%%%%%%

\section{Introduction}

Elliptic Modules, later \textit{Drinfeld Modules}, were originally introduced in ~\cite{Drinfeld74} as a theoretical tool for solving a special case of the Langlands Conjecture. Since then, the particularly close relationship between rank-two Drinfeld modules and elliptic curves has led to considerable literature exploring whether theoretical results characterizing elliptic curves have Drinfeld module analogies. Schoof's algorithm for point counting on Elliptic curves is one area of particular interest, and though a perfect analogy did not appear to be practical, a number of algorithms for computing the Frobenius trace of a rank-two Drinfeld module were presented in \cite{MuslehSchost}.

In this paper, we attempt to extend these techniques to Drinfeld modules of rank $r > 2$. One of the most critical differences observed in the transition to Drinfeld modules of arbitrary rank is that it becomes much more difficult to provide theoretical guarantees that these algorithms return a correct result. Much of this work is therefore devoted to showing that under certain parameter conditions, we can expect these techniques to succeed with high probability. 

\section{Preliminaries}

\subsection{Elementary Constructions}
To allow for consistency with the notation used in ~\cite{MuslehSchost}, $\F_q$ will denote a finite field of order $q$ a prime power, and $\L \supset \F_q$ shall denote a finite extension of degree $n$. Moreover, $\L$ will come with an ``$A$-field" structure which equips it with an $\F_q$-algebra homomorphism $\gamma$ from $ \L$ to $\F_q[x]$. Then $\ker(\gamma)$ is a prime ideal generated by a monic irreducible polynomial $\mathfrak{p}$ of degree $m$, and $n = m \cdot s$. We refer to $\mathfrak{p}$ as the $\mathbb{F}_q[x]$-characteristic of $\L$.

The ring of \textit{skew-polynomials} is defined as the set:

\begin{align*}
    \L\{\tau\} = \bigg\{ \sum_{i=0}^su_i\tau^i \textnormal{ | } s \in \N, u_i \in \L \bigg\}.
\end{align*}

subject to the relation $\tau a = \pi(a) \tau$. The \textit{skew-degree} of a skew-polynomial is the integer $s$ in the preceding definition. $\L\{\tau\}$ is a right Euclidean domain, and so given any skew-polynomial $\psi$ we let $\L\{ \tau \}/\psi$ denote the right quotient ring. 
There is a natural interpretation of skew-polynomials as elements of $\textnormal{End}_{\F_q}[\L]$ under the following map:

\begin{equation*}
    \iota : \sum_{i=0}^su_i\tau^i \mapsto \sum_{i=0}^su_i\pi^i 
\end{equation*}

with $\pi^0$ denoting the identity mapping on $\L$.

%Let $[n] = \{1, \ldots, n \} $, and let $P_k$ acting on a set $s$ denote the set of subsets of $s$ of size $k$. If $s$ is an element of $P_k([n])$, and $M$ is an $n \times m$ matrix, let $M_s$ be the $(n - k) \times m$ matrix given by deleting rows corresponding to the indices in $s$. We define the square determinant $\sqdet M$ of an $n \times m$ matrix with $n \geq m$ to be:

%\begin{equation*}
%    \sqdet M = \sum_{s \in P_{n-m}([n])} \det M_{s}
%\end{equation*}

%We note that if $\sqdet M$ is non-zero the overdetermined system given by $Mx = b$ either has a unique solution or is inconsistent for any choice of $b$.

\subsection{Algorithmic Background}

For all of our cost analysis, we will assume that all elementary operations on polynomials of degree at most $d$, which includes addition, multiplication, division and extended GCD, over $\F_q[x]$ can be completed in time $(d \log q)^{1 + o(1)}$ \cite{vonzurgathen_gerhard_2013}. We let $\omega$ denote an exponent such that any two $s\times s$ matrices over any ring can be multiplied with at most $O(s^{\omega})$ ring operations. 

\subsubsection{Modular Composition} 
Given $F,G,H \in \F_q[x]$, with $\deg H = d$ and $\deg F, \deg G \leq d$, the operation \textit{modular composition} computes $F(G) \mod H$. Let $\theta$ denote an exponent such that degree $d$ modular composition in $\F_q[x]$ can be done in $(d^{\theta} \log q)^{1 + o(1)}$ bit operations. The algorithm due to Kedlaya and Umans in \cite{kedlaya_umans} allows $\theta = 1 + \epsilon$ for any $\epsilon > 0$, but there do not appear to be any practical implementations of this algorithm to our knowledge. Modular composition allows for computation of Frobenius powers $t^{q^i}$ for any $t \in \L$ via the composition $t(x^{q^i}) \mod H$ where $H$ is a degree $n$ polynomial such that $\L$ can be represented as the residues $\F_q[x]/H$, 

\subsubsection{Skew-Polynomial Multiplication and Division} 

To our knowledge there currently exists two skew-polynomial multiplication algorithms either of which may be the most efficient depending on the parameter cases being dealt with and the details of the cost model involved. The algorithm given in \cite{PuchingerW15} can multiply two degree $d$ skew-polynomials using $(d^{(\omega + 1)/2}n^{\theta} \log q)^{1 + o(1)}$ bit operations. A result due to Caruso and Le Borgne \cite{CaLe17} shows that up to logarithmic factors, left and right division of skew-polynomials belong to the same complexity class as multiplication.

\subsection{Drinfeld Modules}

We are now ready to define the notion of a Drinfeld module.

\begin{definition}
A \emph{Drinfeld module} over $(\L, \gamma)$ is a ring homomorphism $\phi: \F_q[x] \to \L\{ \tau \}$ such that:
\end{definition}

\begin{equation}\label{definition}
    \phi_x = \Delta_r \tau^r + \ldots + \Delta_1 \tau + \gamma(x) 
    \hspace{5mm} \Delta_i \in \L
\end{equation}

The largest index $r$ such that $\Delta_r \neq 0$ is the \textit{rank} of the Drinfeld module. A Drinfeld module over $\F_q[x]$ may be specified using the coefficients of $\phi_x$ given in (\ref{definition}). A morphism of Drinfeld modules $\sigma: \phi \to \psi$ is an element $\sigma$ of $L\{ \tau \}$ such that $\sigma \phi_a = \psi_a \sigma$ for all $a \in \F_q[x]$. The \textit{minimal polynomial} $M_{\phi}$ of the Frobenius endomorphism $\tau^n$ of a Drinfeld module is the monic polynomial of lowest degree $r_m \leq r$, $M_{\phi} = T^{r_m} - \sum_{i=0}^{r_m - 1} a_iT^i \in \F_q[x][T]$ such that:

\begin{equation}\label{characteristic}
    \tau^{n r_m} - \phi_{a_{r_m-1}}\tau^{n(r_m -1)} - \ldots - \phi_{a_1} \tau^n - \phi_{a_0} = 0, 
\end{equation}

where $\deg a_i \leq \frac{(r - i)n}{r}$. The characterization given in \cite{assong2020explicit} shows that $r = r_m \cdot r'$ with $ r' | s$. We may refer to this as simply the minimal polynomial of the Drinfeld module. We will be primarily concerned with algorithms computing the coefficients $\{ a_i \}_{i=0}^{r-1}$. An explicit formula to compute $a_0$ is given in \cite{GaPa18}:

\begin{equation}\label{norm}
    a_0 = (-1)^{r + n(r+1)}N_{\F_{\mathfrak{p}}/\F_q}(\Delta_r)^{-1} \mathfrak{p}^{ \frac{n r_m}{ r m}}
\end{equation}

\subsection{Revisiting the Rank-Two Case}

The problem of computing characteristic polynomials of rank-2 Drinfeld modules was first extensively studied in \cite{gekeler}, where two algorithms were proposed. The first is a generic algorithm that computes the minimal polynomial of any rank 2 Drinfeld module by using a recurrence to compute the coefficients of $\phi_{x^i}$ for $i \leq \frac{n}{2}$, which allows the computation of the coefficients of $a_1$ using (\ref{characteristic}), and takes $\tilde{O}(n^3)$ bit operations. The second algorithm uses an analogue of the Hasse invariant $H$ for Drinfeld modules defined as the coefficient of $\tau^m$ in $\phi(\frakp)$. When $\L = \F_{\frak{p}}$, $a_1$ can be written as

\begin{equation*}
    a_1 = (-1)^n N_{\L/\F_q}(\Delta_2)^{-1}H
\end{equation*}

$H$ itself may be computed using a single linear recurrence which allows an $\tilde{O}(n^{2 + \epsilon} \log^{1 + o(1)} q)$ bit complexity

\subsubsection{Schoof-Like Algorithm} Inspired by Schoof's algorithm for elliptic curves, this approach aims to compute $a_1 \bmod \psi$ for irreducibles $\psi \in \F_q[x]$ and reconstruct $a_1$ using the Chinese Remainder Theorem. We may reduce equation (\ref{characteristic}) on the right by $\phi_{\psi}$ to obtain

\begin{equation*}
    \tau^{2n} + (\phi_{a_1 \textnormal{ mod } \psi}) \tau^n + \phi_{a_0 \textnormal{ mod } \psi} = 0
\end{equation*}

Choosing irreducibles $\psi_1, \ldots, \psi_s$ with $\sum_{i=1}^s \deg \psi_i > n/2$ and $\deg \psi_i \in O(\log n)$ for all $i$, the algorithm computes $\tau^n \mod \phi_{\psi_i}$ and $\phi_{a_0 \textnormal{ mod } \psi_i}$ and solves for $a_1 \mod \psi_i$ using the characteristic equation. Performing this operation $s$ times and reconstructing $a_1$ runs at an overall cost $(n^2\log q + n \log^2 q)^{(1 + o(1))}$.

\subsubsection{Minimal Polynomials from Hankel Systems} Starting with the characteristic equation $\phi_{a_1} = -\tau^{2n} - \phi_{a_0}$, set $\phi_{a_1} := \sum_{i=0}^{\lfloor n/2 \rfloor} c_i\phi_{x}^i$ and $r := -\tau^{2n} - \phi_{a_0}$. For a uniformly random $\alpha \in \L$, apply operators $\phi_{a_1}$ and $r$:

\begin{equation*}
    \sum_{i=0}^{\lfloor n/2 \rfloor} c_i\phi_{x}^i(\alpha) = r(\alpha)
\end{equation*}

For any $j > 0$, apply $\phi_x^j$ to the preceding expression

\begin{equation*}
    \sum_{i=0}^{\lfloor n/2 \rfloor} c_i\phi_{x}^{i +j}(\alpha) = \phi_x^jr(\alpha)
\end{equation*}

After applying a uniformly random projection $\ell : \L \to \F_q$, each such expression yields a row in the $\nu \times \nu$ Hankel system over $\F_q$:

\begin{equation*}
    \begin{bmatrix}
    \ell(\alpha) & \ldots & \ell(\phi_x^{\nu - 1}(\alpha)) \\
    \ell(\phi_x(\alpha)) & \ldots & \ell(\phi_x^{\nu}(\alpha)) \\
    \vdots & & \vdots \\
    \ell(\phi_x^{\nu - 1}(\alpha) & \ldots & \ell(\phi_x^{2\nu - 2}(\alpha)
    \end{bmatrix}
    \begin{bmatrix}c_0 \\ \vdots \\ c_{\nu - 1} \end{bmatrix} = 
    \begin{bmatrix} \ell (r(\alpha)) \\ \vdots \\ \ell(\phi_x^{\nu - 1}r(\alpha)) \end{bmatrix}
\end{equation*}

If $n$ is even, we may not be able to determine $c_{\lfloor n/2 \rfloor}$, in which case we let $\nu = \lfloor n/2 \rfloor$ and compute $c_{\lfloor n/2 \rfloor}$ via the formula
\begin{equation*}
    c_{\lfloor n/2 \rfloor} = \textnormal{Tr}_{\F_{q^2}/\F_{q}}(\textnormal{N}_{\L/\F_{q^2}}(\Delta)^{-1}),
\end{equation*}
here $\F_{q^2}$ is the unique quadratic extension of $\F$ contained in $\L$, and $\textnormal{Tr}$ and $\textnormal{N}$ are the field trace and norm respectively. One can compute up to $2n$ elements of the sequence $\{ \ell \phi_x^i(\alpha)\}$ using $(n^2 \log^2q)^{1 + o(1)}$ bit operations. $r(\alpha)$ and the resulting sequence $\ell(\phi_{x^i}(r(\alpha)))$ for $i \leq \nu -1$ can be computed using at most $(n^2 \log^2q)^{1 + o(1)}$ bit operations. Finally, the resulting Hankel system can be solved in $(n^2 \log q)^{1 + o(1)}$ bit operations, allowing for an overall runtime of $(n^2 \log^2q)^{1 + o(1)}$. 

%\begin{equation*}
%    \phi_x^i\tau^{2n} + \phi_
%\end{equation*}

\section{Minimal Polynomials in Higher Ranks}

\subsection{Previous Work}

The following approach is due to \cite{GaraiPapikian} and can be used when $\L = \F_{\frak{p}}$. Rewrite the expression for the minimal polynomial as $f_i' + f_i = 0$ where

\[ f_i = \phi_{a_{r-i}}\tau^{n (r-i)} + \phi_{a_{r - i -1 }}\tau^{n (r-i-1)} + \ldots + \phi_{a_0} \]

\[ f_i' = \tau^{nr} + \phi_{a_{r -1 }}\tau^{n (r-1)} + \ldots + \phi_{a_{r-i + 1}}\tau^{n(r - i + 1)}. \]

The term of skew degree $n(r - i + 1)$ in $f_i'$ is  $\gamma(a_{r-i+1}) \tau^{n(r-i + 1)}$; consequently each $\gamma(a_{r-i+1})$ can be computed by determining the $n(r - i + 1)$ skew-degree term of $f_i$.

The bit-complexity of computing all of $\phi_x, \phi_{x^2}, \ldots, \phi_{x^n}$ is \newline
$\Theta(r\log(r)n^3\log(q))$. Computing each $f_i$ is linear in the degree of the term being added, which is $O(rn)$, which is done for $O(r)$ iterations, so the overall cost is $O((r^2n + r\log(r)n^3)\log(q) )$.


\subsection{Schoof-like Algorithms}

Our main results for this section is the following theorem.

\begin{theorem}\label{schooflike}
There exists a Monte Carlo algorithm such that when $\gcd(n,r) = 1$ or $r = 3$, then for a generic choice of rank $r$ Drinfeld module $\phi$ over a degree $n$ extension $L$ of  $\F_q$, the algorithm computes the minimal polynomial of $\phi$ with bit complexity $r^{(\omega+3)/2}(n^2 \log q)^{1 + o(1)}$ and returns a correct result when $q > Cnr^2$, for some constant $C > 2$, with probability greater than $\frac{1}{2}$.
\end{theorem}

Recall the \textit{minimal polynomial} of $\phi$ is the unique monic polynomial $T^{r_m} - a_{r_m-1}T^{r_m-1} - \ldots - a_1T - a_0$ of least degree such that:

\begin{equation} \label{charpoly1}
 \tau^{nr_m} - \phi_{a_{r_m-1}}\tau^{n(r_m-1)} - \ldots - \phi_{a_1} \tau^n - \phi_{a_0} = 0
\end{equation}


Suppose we have an estimate $\hatr$ of $r_m$. The goal is to compute each $a_i$ modulo a set of polynomials which can then be used to reconstruct the true coefficients. As in the rank 2 case, when $q \gg nr$, linear polynomials of the form $x - e$ suffice and equation (\ref{charpoly1}) becomes

\begin{equation}\label{linsys}
\tau^{n\hatr} - a_{\hatr-1}(e)\tau^{n(\hatr-1)} - \ldots - a_1(e)\tau^n - a_0(e)  = 0 \mod \phi_{x - e}.
\end{equation}

$\L\{\tau \}/\phi_{x-e}$ is an $r$-dimensional vector space over $L$ spanned by the computational basis $\{1, \tau, \ldots, \tau^{r-1}\}$. Writing (\ref{linsys}) as a linear system

\begin{equation}\label{system1} \begin{bmatrix} \tau^{n(\hatr-1)} & \ldots & \tau^{n} \end{bmatrix} \begin{bmatrix} a_{\hatr-1}(e) \\ \vdots \\ a_1(e) \end{bmatrix} = \begin{bmatrix}  -a_0(e) \\ 0 \\ \vdots \\ 0  \end{bmatrix} + \tau^{n\hatr}  \mod \phi_{x-e}.
\end{equation}

Let $ B(\psi) = \begin{bmatrix} \tau^{n(\hatr-1)} & \ldots & \tau^{n} \end{bmatrix} \mod \psi$ for $\psi \in \L\{\tau\}$. The unique solvability of the system encoded by $B(\phi_{x-e})$ can be characterized using the following lemma.

%Let $B_{\phi}^{n,j} := \begin{bmatrix} \tau^{n(j-1)} & \ldots & \tau^{n} \end{bmatrix} \bmod \phi_{x-e}$ be the $r \times (j -1)$ system encoding the coefficients of the $r$-dimensional vector $\tau^{ni} \mod \phi_{x -e}$ in the computational basis. The goal is to show the system given in $(\ref{system1})$ has a unique solution with high probability when $q \gg nr$ by applying the Schwartz-Zippel lemma to the polynomial in the coefficients of the Drinfeld module given by $\det B_{\phi}^{n,\hatr}$, where $B_{\phi}^{n,\hatr}$ is truncated to the first $\hatr - 1$ rows, as $\det B_{\phi}^{n,\hatr} \neq 0$ provides a sufficient condition for solvability.

%Let $ \Delta_i = \sum_{j=0}^{n-1}\alpha_{i,j}x^j$ for $i > 0$ and $ (\gamma(x) - e) = \sum_{j=0}^{n-1}\alpha_{0,j}x^j$. The goal of the next lemma is to bound the degree 

\begin{lemma}\label{detdeg}
Suppose $Cn^2r < q$ for a constant $C>2$. Further suppose that there exists at least one skew-polynomial modulus $\psi$ of skew-degree $r$ such that the system in (\ref{system1}) is uniquely solvable. Then for generic choices of Drinfeld coefficients $\Delta_i$ and evaluation point $e$, the system given in $(\ref{system1})$ is uniquely solvable.
\end{lemma}

\begin{proof}
We consider general skew-polynomial moduli $\psi = \sum_{i=0}^{r} \Delta_i \tau^i$ for equation $(\ref{system1})$ rather than the specific case of $\Delta_0 = x - e$. Generically, $\hatr = r$, and for $n,r \geq 3$ we can set $\mu = n(\hatr - 1)$, $\reduc = \mu - r$, $\omega = \mu - n > 0$. We can characterize the linear independence of the set $\{\tau^{ni} \}_{i =1}^{\hatr}$ as follows: suppose to the contrary that

\begin{equation}\label{mult}
    \sum_{i=1}^{\hatr - 1} \beta_i \tau^{ni} = P \phi_{x - e}
\end{equation}

with $P = \sum_{i= n}^{\reduc} p_ix^i \neq 0$. The choice of $P$ is constrained by the requirement that the coefficient of $\tau^{j}$ in the product is 0 if $n$ doesn't divide $j$. The linear system representing the product $P\phi_{x-e}$ can be encoded as an $ (\omega + 1) \times (\omega - r + 1)$ matrix $M$ over $\L$

\begin{equation}
    \label{productsystem}
    \begin{bmatrix}
    \Delta_{r}^{q^{\reduc}} & 0 & 0 & \ldots & 0 & \ldots & 0 \\
    \Delta_{r - 1}^{q^{\reduc}} & \Delta_{r}^{q^{\reduc - 1}} & 0 & \ldots & 0 & \ldots & 0 \\
    \vdots & \vdots & \vdots & \ddots & \vdots & \ddots & \vdots \\
    \Delta_{0}^{q^{\reduc}} & \Delta_{1}^{q^{\reduc - 1}} & \Delta_{2}^{q^{\reduc - 2}} & \ldots & \Delta_{r}^{q^{\reduc - r}} & \ldots & 0 \\
    0 & \Delta_{0}^{q^{\reduc - 1}} & \Delta_{1}^{q^{\reduc - 2}} & \ldots & \Delta_{r-1}^{q^{\lambda - r}} & \ldots & 0 \\
    \vdots & \vdots & \vdots & \ddots & \vdots & \ddots & \vdots \\
    0 & 0 & 0 & \ldots & 0 & \ldots & \Delta_0^{q^{n}}
    
    
    \end{bmatrix}
    \begin{bmatrix}
    p_{\reduc} \\ p_{\reduc - 1} \\ \vdots \\ p_{n}
    \end{bmatrix}
    = 
    \begin{bmatrix}
    \beta_{\hatr-1} \\ 0 \\ \vdots \\ 0 \\ \beta_{1}
    \end{bmatrix}
\end{equation}

Deleting $\hatr - 1$ rows corresponding to the free parameters $\beta_i$ leaves the above as a well determined $ (\omega - \hatr + 2) \times (\omega - r + 1)$ linear system. 

\begin{equation}
    \label{reduced}
    \begin{bmatrix}
    \Delta_{r - 1}^{q^{\reduc}} & \Delta_{r}^{q^{\reduc - 1}} & 0 & \ldots & 0 & \ldots & 0 \\
    \vdots & \vdots & \vdots & \ddots & \vdots & \ddots & \vdots \\
    \Delta_{0}^{q^{\reduc}} & \Delta_{1}^{q^{\reduc - 1}} & \Delta_{2}^{q^{\reduc - 2}} & \ldots & \Delta_{r}^{q^{\reduc - r}} & \ldots & 0 \\
    0 & \Delta_{0}^{q^{\reduc - 1}} & \Delta_{1}^{q^{\reduc - 2}} & \ldots & \Delta_{r-1}^{q^{\lambda - r}} & \ldots & 0 \\
    \vdots & \vdots & \vdots & \ddots & \vdots & \ddots & \vdots \\
    
    
    \end{bmatrix}
    \begin{bmatrix}
    p_{\reduc} \\ p_{\reduc - 1} \\ \vdots \\ p_{n}
    \end{bmatrix}
    = 
    \begin{bmatrix}
    0 \\ 0 \\ \vdots \\ 0 \\ 0
    \end{bmatrix}
\end{equation}

Deleting an additional $r - \hatr + 1$ rows, chosen such that the resulting system derived from the modulus $\psi$ is invertible, and showing that for generic choices of Drinfeld module parameters the determinant of this resulting square matrix is non-zero is sufficient to establish $P = 0$. To that end, we ``vectorize" equation $(\ref{productsystem})$: elements $a \in \L$ can be represented by a length $n$ vector $\hat{a}$ with entries in $\F_q$ or by their $n\times n$ multiplication operator $\overline{a}$. Moreover, let $\overline{\sigma}$ denote the $n\times n$ matrix of the Frobenius acting on vectors $\hat{a}$. This allows a rewriting of the truncated system $\overline{M}$

\begin{equation}
    \label{vectorsystem}
    \begin{bmatrix}
    \overline{\sigma}^{\lambda}\overline{\Delta}_{r - 1} & \overline{\sigma}^{\lambda-1}\overline{\Delta}_{r} & 0 & \ldots & 0 & \ldots & 0 \\
    \vdots & \vdots & \vdots & \ddots & \vdots & \ddots & \vdots \\
    \overline{\sigma}^{\lambda}\overline{\Delta}_{0} & \overline{\sigma}^{\lambda -1}\overline{\Delta}_{1} & \overline{\sigma}^{\lambda - 2}\overline{\Delta}_{2} & \ldots & \overline{\sigma}^{\lambda - r}\overline{\Delta}_{r} & \ldots & 0 \\
    0 & \overline{\sigma}^{\lambda - 1}\overline{\Delta}_{0} & \overline{\sigma}^{\lambda - 2}\overline{\Delta}_{1} & \ldots & \overline{\sigma}^{\lambda - r}\overline{\Delta}_{r-1} & \ldots & 0 \\
    \vdots & \vdots & \vdots & \ddots & \vdots & \ddots & \vdots 
    
    
    \end{bmatrix}
    \begin{bmatrix}
    \hat{p}_{\reduc} \\ \hat{p}_{\reduc - 1} \\ \vdots \\ \hat{p}_{n}
    \end{bmatrix}
    = 
    \begin{bmatrix}
     \hat{0} \\ \vdots \\ \hat{0} 
    \end{bmatrix}
\end{equation}

This system is parametrized by the $\F_q$ coefficients of $\Delta_i = \sum_{j = 0}^{n-1} \alpha_{i,j}x^j$. The resulting vectorized system is an  $ n(\omega - r + 1) \times n(\omega - r + 1)$ matrix whose entries are degree 1 polynomials in the variables $\alpha_{i,j}$; therefore when the determinant is not exactly the zero polynomial in the $\alpha_{i,j}$, then it has degree $n(\omega - r + 1) = O(n^2r)$. This can't be the case since the system must be invertible for at least one choice of $\alpha_{i,j}$ corresponding to the choice of skew-polynomial $\psi$ for which $B(\psi)$ is uniquely solvable. Applying the Schwartz-Zippel lemma to the determinant:

%Set $\alpha_i = \frac{\Delta_i}{\Delta_r}$ with coefficients $\alpha_i := \sum_{j=0}^{n-1}\alpha_{i,j}x^j$. Furthermore, let $\overline{\alpha_i}$ be the $n\times n$ matrix of the action of left multiplication by $\alpha_i$ on $\L/\F_q$, and let $\overline{\tau}$ denote the matrix representation of the Frobenius. If $\sum_{i=0}^{r-1}c_i \tau^i$ is any skew polynomial, the action of left multiplication by $\tau$ on $\L[\tau]/\phi_{x-e}$ is given by the following matrices:

%\begin{equation*}
%M :=  \begin{bmatrix}\overline{\alpha}_{r-1} & I & 0 & \ldots & 0 \\ \overline{\alpha}_{r-2} & 0 & I & \ldots & 0 \\ \vdots & \vdots & \vdots & \ddots & \vdots \\ \overline{\alpha}_{0} & 0 & 0 & \ldots & I \end{bmatrix} \hspace{4mm} S := \begin{bmatrix} \overline{\tau} & & \\ & \ddots & \\ & & \overline{\tau} \end{bmatrix}
%\end{equation*}

%\begin{equation*}
%\begin{bmatrix}\overline{\alpha}_{r-1} & I & 0 & \ldots & 0 \\ \overline{\alpha}_{r-2} & 0 & I & \ldots & 0 \\ \vdots & \vdots & \vdots & \ddots & \vdots \\ \overline{\alpha}_{0} & 0 & 0 & \ldots & I \end{bmatrix} \begin{bmatrix} \overline{\tau} & & \\ & \ddots & \\ & & \overline{\tau}\end{bmatrix} \begin{bmatrix} \overline{c}_{r-1} \\ \vdots \\ \overline{c}_{0} \end{bmatrix} = \begin{bmatrix} \overline{\alpha}_{r-1} \overline{\tau}(\overline{c}_{r-1}) + \overline{\tau}(\overline{c}_{r-2}) \\ \overline{\alpha}_{r-2} \overline{\tau} \overline{c}_{r-1} + \overline{\tau} \overline{c}_{r-3} \\ \vdots \\ \overline{\alpha}_{0} \overline{\tau}(\overline{c}_{r-1})  \end{bmatrix}
%\end{equation*}

%with each $\overline{c}_i \in \L$ being written as a size $n$ vector over $\mathbb{F}_q$. Then $\tau^{in} \mod \phi_{x-e}$ is given by 

%\begin{equation*}
%(MS)^{in-r} \begin{bmatrix} \overline{1} \\ 0 \\ \vdots \\ 0 \end{bmatrix}.
%\end{equation*}

%The entries of this expression have degree $in - r$ in the coefficients $\alpha_{i,j}$, and the total degree of any term in the square determinant is $2\sum_{i=1}^{r-1} (in - r) \leq 2nr^2$.

\begin{equation*}
    \Pr\Big[\det \overline{M} = 0\Big] \leq \frac{n^2r}{q} < \frac{1}{C}.
\end{equation*}

\end{proof}

%If $\sqdet B_{\psi}^{n,r} \neq 0 $, then for generic choices of base field element $e$ and coefficients of $\phi_x$, the Schwartz-Zippel lemma implies

% $\{\alpha_{i,j}\}_{1 \leq i \leq r-1}^{0 \leq j \leq n-1}$, the Schwartz-Zippel lemma implies



The remainder of the proof of theorem 1 is classifying pairs of parameters $n,r$ for which there exists at least one choice of modulus $\psi \in \sring$ that guarantees unique solvability of $B(\psi)$

\begin{prop}\label{coprime}
Suppose $\gcd(n,r) = 1$. There exists a Drinfeld module $\phi_x$ such that $ B(\phi_x)$ is uniquely solvable.  %e $\psi = \tau^n - 1$.
\end{prop}
\begin{proof}
Let $\phi_x = \tau^n - 1$. For any $e \neq 1$, $\tau^{ni} = (e + 1)^{\lfloor ni/r \rfloor} \tau^{ni \bmod r}$. Since $\gcd(n,r) = 1$, the set $\{ n \mod r, 2n \mod r, \ldots, rn \mod r\}$ is a reordering of $\{0, 1, \ldots, r-1\}$ and so the elements $\tau^{ni}$ form an $\L$ basis of $\L\{ \tau\}/\phi_{x - e}$.
\end{proof}



\begin{prop}\label{rank3}
If $r = 3$, then for any $n$ there exists a skew-polynomial $\psi \in \L\{\tau\}$ such that $B(\psi)$ is uniquely solvable.
\end{prop}
\begin{proof}
The notion of an \textit{Azumaya algebra} and their connection to skew polynomials was previously studied in \cite{Ikehata1984AzumayaAA} and \cite{skewfactor}. In particular, we are interested in a distinguished map sending an Azumaya algebra to its center, the \textit{reduced norm}, which induces a map $\cN: \L\{\tau\} \to \comm$ while satisfying the following constraints.

\begin{enumerate}
    \item $\cN(\psi_1 \psi_2) = \cN(\psi_1) \cN(\psi_2)$
    \item $\deg \cN(\psi) \leq n \deg \psi$
    \item $\cN(\psi) = \psi^n$ if $\psi \in \L[\tau^n]$
    \item $\psi$ is irreducible in $\sring$ if and only if $\cN(\psi)$ is irreducible in $\F_q[\tau^n]$
\end{enumerate}

For details on the construction of the map and the proof of item (4) we refer to \cite{skewfactor}. Suppose $\tau^{2n} - \lambda \tau^n = 0$ for some $\lambda \in \F_q$. Then $\tau^{n} - \lambda = 0 \mod \psi$, and we must have a skew-polynomial $D$ such that $\tau^n - \lambda = D \psi$. Taking reduced norms of both sides leads to

\begin{equation*}
(\tau^n - \lambda)^n = \cN(D)\cN(\psi)
\end{equation*}

The following observation shows that we may always find an irreducible degree 3 skew-polynomial in $\L\{\tau\}$. If there are none, then the mapping of triples $(\alpha, \beta, \gamma) \to (a, b, c)$ from $\L^3$ implied by $(\tau^2 + \alpha \tau + \beta)(\tau + \gamma) = \tau^2 + a \tau^2 + b\tau + c$ must be bijective. However both $(3, 2, 1)$ and $(2, 1, 2)$ map to $(4, 5, 2)$ and that this holds in any characteristic.  If we choose $\psi$ to be any degree 3 skew-polynomial irreducible in $\L\{\tau \}$, which must exist if $\L$ is finite, item (4) implies $\cN(\psi)$ is an irreducible central skew-polynomial dividing $(\tau^n - \lambda)^n$ and so $\cN(\psi) = \tau^n - \lambda$. But we must then have $n^2 = \deg \cN(D) + \deg \cN(\psi) \leq (n-3)n + n = n^2 - 2n < n^2$, giving a contradiction.

\end{proof}

\subsection{Limitations}

One of the clearest limitations of this approach is the large base field requirement $q > Cn^2 r$. Unlike the case $r = 2$, working with arbitrary degree irreducibles in $\F_q[x]$ in place of degree 1 polynomials does not appear to repair the issues discussed in lemma (\ref{detdeg}).

Lemma (\ref{detdeg}) also appears limited to only the parameter cases given in propositions (\ref{coprime}) and (\ref{rank3}). As will be seen in the following example, there exist parameter choices for $n, r,$ and Drinfeld module $\phi_x$ outside these cases where the algorithm will never work.

\begin{example}\label{examplefail}
Let $n = 3$, $r =3$, $\phi_x = \tau^3 + x$. For $(\ref{system1})$ to be uniquely solvable for some $e$, whenever $\alpha_1 \tau^3 + \alpha_2 \tau^6 = P (\tau^3 + x - e)$ we must have $P = 0$. However for any $e$ setting $\alpha_1 = 1$, $\alpha_2 = x - e $ has the non-zero solution $P = \tau^3$.  
\end{example}

It is possible to detect failure cases such as those in the preceding example by checking dimensionality of the kernel of the linear system in $(\ref{reduced})$.  



\subsubsection{Complexity.}

The algorithm can be summarized with the following routine:

%\textit{Subroutine 1}
\begin{enumerate}
    \item Compute $\tau^{ni} \mod \phi_{x-e}$ in the computational basis for $\lfloor\frac{r-1}{r}n \rfloor$ distinct $e \in \F_q$ and $1 \leq i \leq r$
    \item Solve the $r \times (\hatr-1)$ system given in (\ref{system1}) for each choice of $e$. If the system $B(\phi_{x-e})$ is uniquely solvable, apply a uniformly random projection $\ell : \L \to \F_q$ to the system and solve the system over $\F_q$.
    \item Interpolate $\{a_i\}_{i=1}^{\hatr-1}$ using the $\lfloor\frac{r-1}{r}n \rfloor$ evaluation points $\{a_i(e)\}_{i=1}^{r-1}$ computed in step (2). 
\end{enumerate}

The bottleneck of subroutine 1 is step (1). Algorithms for skew-polynomial multiplication can be leveraged to compute the entire collection of $\tau^{ni} \mod \phi_{x - e}$ in $r^{(\omega+3)/2}(n \log q)^{1 + o(1)}$ bit operations for a single $e$, with an overall contribution to the algorithm of $r^{(\omega+3)/2}(n^2 \log q)^{1 + o(1)}$. Step (2) adds $r^{\omega}n\log q$ bit operations, and (3) is of negligible cost. 

A generic choice of projection $\ell$ will preserve unique solvability of $B(\phi_{x-e})$ with probability at least $1 - \frac{\hatr}{q} \geq 1 - \frac{1}{Cn^2} $. This holds for $O(n)$ independent choices of projection with probability at least $(1 - \frac{1}{Cn^2})^n \to e^{-1/(C)} \to 1$ for large $n, C$. The remaining probabilistic element is the likelihood of genericity for the choice of Drinfeld module, base field element pair $(\phi, e)$ for $n$ choices of $e$, which happens with probability $(1 - \frac{1}{Cn})^{n}$ and which can be made arbitrarily close to 1. This completes the argument for theorem 1.

When working with the prime field case $n = m$ or when $\gcd(r, s) = 1$, the minimal polynomial is known to have degree $r$ and the routine need only be invoked once. For the general case, including parameter choices not covered by theorem \ref{schooflike}, we speculate it may be possible to invoke the algorithm for choices of $\hat{r}$ such that $r = \hat{r} r'$ with $r' | s$ and invoking probabilistic polynomial testing to verify that the output for a given $\hat{r}$ is in fact an annihilating polynomial for $\tau^n$; taking the polynomial corresponding to the least such $\hat{r}$.

%\begin{enumerate}
%    \item Initialize $r_{upper} \leftarrow r$, $r_{lower} \leftarrow 1$.
%    \item If $r_{upper} = r_{lower}$, return $r_{upper}$. Otherwise set $\hatr \leftarrow \frac{r_{upper} + r_{lower}}{2}$.
%    \item Run Subroutine 1 on the estimate $\hatr$.
%    \item If the system in (\ref{system1}) is inconsistent, $r_{lower} \leftarrow \hatr$ and go to step 2.
%    \item If the system in (\ref{system1}) returns a solution, use polynomial identity testing to determine if $\tau^{n \hatr} - \sum_{i = 0}^{\hatr - 1} \phi_{a_i}\tau^{ni} = 0$. If this equation holds, set $r_{upper} \leftarrow \hatr$, otherwise $r_{lower} \leftarrow \hatr$. Go to step 2.
    
%\end{enumerate}





\section{From Hankel Systems}
Inspired by the Monte Carlo Algorithm given in $\cite{MuslehSchost}$, we will propose an additional algorithm which doesn't depend on the genericity of the chosen Drinfeld module required by the Zippel-Schwartz criteria, and therefore may be usable when the algorithm of theorem \ref{schooflike} fails, and possesses a similar complexity.
\begin{theorem}\label{hankel}
Let $\phi$ be a rank $r$ Drinfeld module over a degree $n$ extension $L/\F_q$ such that $\gcd(n, r) = 1$. There is a Monte-Carlo algorithm computing the minimal polynomial of $\phi_x$ with bit complexity $r^{3}n^2 \log q$ and which returns a correct result with probability greater than $\frac{1}{2}$ .
\end{theorem}

 We can organize the characteristic equation into a Hankel system.

\begin{align*}
\sum_{i=1}^{r - 1} \phi_{a_i}\tau^{ni} = &  \tau^{nr} - \phi_{a_0}
%\end{equation*}
%\begin{equation}
%\label{chareq}
\\
\sum_{i=1}^{r - 1} \sum_{j=0}^{n(r-i)/r}a_{i,j}\phi_{x}^j\tau^{ni} = & \tau^{nr} - \phi_{a_0}
\end{align*}

Consider the operator Hankel blocks
\begin{align*}
H_{u,v} = & \begin{bmatrix}
\phi_x^u\tau^{n} & \phi_x^u\tau^{2n} & \ldots & \phi_x^u\tau^{n (r - 1 - \lfloor(ur/n) \rfloor)} \\
\phi_x^u\tau^{2n} & \phi_x^u\tau^{3n} & \ldots & \phi_x^u\tau^{(1 + r - 1 - \lfloor(ur/n) \rfloor)n} \\ \vdots & \vdots & \ddots & \vdots \\ \phi_x^u\tau^{vn} & \phi_x^u\tau^{(1+v)n} & \ldots & \phi_x^u\tau^{(v +  r - 1 - \lfloor(ur/n) \rfloor) n}
\end{bmatrix}
\\
%\end{equation*}
%\begin{equation*}
    G_{v} = &\begin{bmatrix}
    \tau^{nr} - \phi_{a_0} & \tau^{n(r+1)} - \phi_{a_0}\tau^n & \ldots \tau^{n(r+v -1)} - \phi_{a_0}\tau^{n(v-1)}
    \end{bmatrix}^T
\end{align*}

Define
\begin{align*}
%\begin{equation*}
H:= & \begin{bmatrix}
H_{0,v_1} & H_{1, v_1} & \ldots & H_{\lfloor n (r- 1)/r \rfloor, v_1} \\ \vdots & \vdots & \ddots & \vdots \\ H_{0,v_k} & H_{1, v_k} & \ldots & H_{\lfloor n (r-1)/r \rfloor, v_k}
\end{bmatrix}\\
G:= & \begin{bmatrix}
G_{v_1} \\ \vdots \\ G_{v_k}
\end{bmatrix}
%\end{equation*}
\end{align*}

We then obtain the system

\begin{equation*}
H
\begin{bmatrix}
a_{1,0} \\ a_{2,0} \\ \vdots \\ a_{1,1} \\ a_{2,1} \\ \vdots \\ a_{1, \lfloor n (r-1)/r \rfloor}
\end{bmatrix} = G
\end{equation*}

Lifting to an extension $K$ over $L$, the goal is to choose a set of evaluation points $\alpha_1, \ldots, \alpha_k \in K$ such that

\begin{equation*}
    H(\alpha_1, \ldots, \alpha_k) := \begin{bmatrix}
H_{0,v_1}(\alpha_1) & H_{1, v_1}(\alpha_1) & \ldots & H_{\lfloor n (r-1)/r \rfloor, v_1}(\alpha_1) \\ \vdots & \vdots & \ddots & \vdots \\ H_{0,v_k}(\alpha_k) & H_{1, v_k}(\alpha_k) & \ldots & H_{\lfloor n (r-1)/r\rfloor, v_k}(\alpha_k)
\end{bmatrix}
\end{equation*}

is non-singular. Note that $H$ is an $ (v_1 + \ldots + v_k) \times \bigg( \sum_{i=1}^{n(r-1)/r} r - 1 - \lfloor(i r/n) \rfloor \bigg)$ system, and that the number of columns is a most $nr$. Let $v_1 = v_2 = \ldots = v_k = r$, $k = n$, and $[K:L] = r$. So $[K:\F_q] = nr$, and observe that if the following two conditions are satisfied:

\begin{enumerate}
    \item The characters $\phi_x^{u}\tau^{ni}$ on $K$ are independent for $1 \leq i \leq r-1$ and $0 \leq u \leq  \frac{r-i}{r} n < n$ 
    \item The set $B = \{\alpha_1, \tau^n(\alpha_1), \ldots , \tau^{n(r - 1)}(\alpha_1), \ldots , \alpha_k, \tau^n(\alpha_k), \ldots , $
    $\tau^{n(r - 1)}(\alpha_k) \}$ is a basis for $K/\F_q$
\end{enumerate} 

Then $H(\alpha_1, \ldots, \alpha_{n})$ must be non-singular. The following lemma summarizes the core of this argument.

\begin{lemma}\label{charindlem}
Let $F/G$ be a finite field extension with $[F:G] = m$ and fix a basis $\alpha_1, \ldots, \alpha_m$ for $F$ over $G$. Let $\sigma_1, \ldots, \sigma_m$ be independent characters over $F$, fixing $G$, and which act linearly on $F$. Then the following matrix is non-singular

\begin{equation*}
 \begin{bmatrix}
\sigma_1(\alpha_1) & \ldots & \sigma_m(\alpha_1) \\ \vdots & & \vdots \\ \sigma_1(\alpha_m) & \ldots & \sigma_m(\alpha_m) 
\end{bmatrix}   
\end{equation*}


\end{lemma}
\begin{proof}
Suppose there exist $\Lambda = \begin{bmatrix} \lambda_1 &  \ldots & \lambda_m \end{bmatrix}^T \in F^m$ with not all $\lambda_i = 0$ such that

\begin{equation*}
 \begin{bmatrix}
\sigma_1(\alpha_1) & \ldots & \sigma_m(\alpha_1) \\ \vdots & & \vdots \\ \sigma_1(\alpha_m) & \ldots & \sigma_m(\alpha_m) \end{bmatrix} \begin{bmatrix} \lambda_1 \\ \vdots \\ \lambda_m \end{bmatrix} = \begin{bmatrix}
0 \\ \vdots \\ 0
\end{bmatrix}
\end{equation*}

Then for any row-wise re-scaling with $\ell_1, \ldots, \ell_m \in G$ we must also have

\begin{equation*}
 \begin{bmatrix}
\ell_1\sigma_1(\alpha_1) & \ldots & \ell_1\sigma_m(\alpha_1) \\ \vdots & & \vdots \\ \ell_m\sigma_1(\alpha_m) & \ldots & \ell_m\sigma_m(\alpha_m) \end{bmatrix} \begin{bmatrix} \lambda_1 \\ \vdots \\ \lambda_m \end{bmatrix} = \begin{bmatrix}
0 \\ \vdots \\ 0
\end{bmatrix}
\end{equation*}

We multiply out the left-hand side and sum the rows to obtain

\begin{equation}\label{violation}
    \sum_{i=1}^m\sum_{j=1}^m\ell_i\lambda_j\sigma_j(\alpha_i) = \sum_{j=1}^m\lambda_j\sigma_j\bigg(\sum_{i=1}^m\ell_i\alpha_i\bigg)  = 0
\end{equation}

Observe now that (\ref{violation}) holds for any $\ell_1, \ldots, \ell_m $, violating independence of characters.

\end{proof}

If $\gcd(n,r) = 1$, the degree of the lead term of $\phi_x^u = C\tau^{ru} + \ldots$ has a unique residue modulo $n$ for distinct choices of $u$, and so independence of $\{ \phi_x^{u}\tau^{ni} \}_{0 \leq u \leq n(r-i)/r }^{1 \leq i < r}$ follows from independence of $id, \tau, \tau^2, \ldots, \tau^{nr-1}$. The values $\alpha_1, \ldots, \alpha_r$ can be constructed from a normal basis for $K/\F_q$ by selecting $\alpha_i = \alpha^{q^i}$ for $0 \leq i \leq r - 1$.

\subsubsection{Complexity.} Summarizing the preceding results, we have the following procedure:

\begin{enumerate}
    \item Fix a degree $r$ extension $K$ of $L$ and a normal basis for $K$ over $\F_q$
    \item Compute the Hankel Blocks $H_{i,r}(\alpha_j)$ as well as the right-hand blocks $G_r(\alpha_j)$
    \item Fix a random projection $\ell: K \to \F_q$ and apply it to the entries of $H, G$
    \item Solve the resulting $O(nr) \times nr$ system over $\F_q$.
\end{enumerate}
Determining a normal basis of size $nr$ of $K/\F_q$ can be done sub-quadratically in the degree of the extension. Constructing the Hankel system above involves computing $\phi_x^{u}\tau^{ni}(\alpha_j)$ for $O(nr^2)$ choices of $0 \leq u < \frac{r-1}{r}n, 1 \leq i \leq r, 0 \leq j \leq r$. The applications of $\tau$ can be applied for $O(1)$ cost on the normal basis, while applications of $\phi_x$ take $O(nr \log q)$ bit operations. The total bit complexity for the entire procedure to compute the block Hankel system is $O(n^2r^3\log q)$. We may again apply the uniformly selected linear projection $\ell$ at an additional cost of $O(n^2r^3\log q)$. Constructing the normal basis and solving the resulting $nr \times nr$ Hankel system consisting of $n$ block columns and $r$ block rows can be done in at most $O(n^2r^2\log q)$.

\subsection{Comparison and Limitations}

Unlike in the Schoof-like case, there is no dependence on the genericity of the choice of Drinfeld module; the algorithm only fails to return a result when a poor choice of projection is made. For cases other than $\gcd(n,r) = 1$, lemma (\ref{charindlem}) can be applied by directly by testing the linear independence of characters $\{ \phi_x^{u}\tau^{ni} \}_{0 \leq u \leq n(r-i)/r }^{1 \leq i < r}$, which can be done probabilistically using polynomial identity testing. 











%%%%%%%%%%%%%%%%%%%%%%%%%%%%%%%%%%%%%%%%%%%%%%%%%%%%%%%%%%%%%%%%%%%%%%%%

\bibliographystyle{ACM-Reference-Format}
\bibliography{drinfeld_charpoly}

\end{document}
